\documentclass{article}
\usepackage{amsmath}
\usepackage{amssymb}
\usepackage{hyperref}
\usepackage[margin=1in]{geometry}
\usepackage{graphicx}
\begin{document}
\noindent Luke Bouma, \today. MESA Stellar Evolution Grid \texttt{README}.

\section{Description of directories}
\begin{itemize}
	\item \texttt{/data/grid\_*}: This directory is a copy of the ``work'' 
	directory that was directly used to run the MESA models. It has all of the 
	inlists for the respective stars. It also has all of MESA's output (photos, 
	output logs, data logs).
	\texttt{01\_make\_inlists.py} is the script used to generate said inlists 
	(and shows the implementation of the logic described in 
	Sec.~\ref{sec:MESA}).
	
	\item \texttt{/src/} This directory has all the scripts that are used to 
	go from MESA output to plots and things that are human-readable.
	Each script has descriptions and usage examples in its header. 
	
	\item \texttt{/results/} This directory has the plots and tables for each 
	stellar grid output by \texttt{src} scripts.
	
	\item \texttt{/doc/} This directory has all the documentation for this 
	project. This includes text files that serve as a lab-notebook, as well as 
	this README.
\end{itemize}

If you would like to reproduce everything:
\begin{itemize}
	\item Install MESA r7503. Follow the MIST readme for installation of their 
	opacity tables and other addons.
	\item Either run the inlists as they're saved in \texttt{data/}, or 
	regenerate them following the logic of \texttt{/data/01\_make\_inlists.py}
	\item After running MESA, and having the stellar models, in \texttt{/src/} 
	run: \texttt{01*}$\rightarrow$\texttt{02*}$\rightarrow$\texttt{03*}.
	\item Look at the results.
\end{itemize}

\begin{align}
Q'_{\star/p} &= Q_0' (P_\mathrm{tide})^\alpha (M)^\beta (M')^\gamma 
(P_\mathrm{spin})^\delta
\label{eq:dissipation_model} \\
Q'_{\star/p} &= 
\begin{cases}
Q_\mathrm{eq}' (P_\mathrm{tide})^{\alpha_\mathrm{eq}} \quad 
&\mathrm{if}\,|P_\mathrm{tide}|<2P_\mathrm{spin}\\
Q_\mathrm{inr}' (P_\mathrm{tide})^{\alpha_\mathrm{inr}} \quad 
&\mathrm{if}\,|P_\mathrm{tide}|\geq2 P_\mathrm{spin}
\end{cases}
\end{align}

\section{MESA stellar evolution grid}	
\label{sec:MESA}
\texttt{POET} computes the orbital evolution of a system with a single star and 
a single planet under the influence of tides.
The tidal dissipation rate in both the planet and the star depend on the
spin period of the body damping the tidal wave 
(Eq.~\protect\ref{eq:dissipation_model}).
To assign a rate of dissipation, we thus need a prescription for the angular 
momentum evolution of both the star and planet.

As the star evolves, its convective envelope loses angular momentum to 
a wind through magnetic braking (Schatzman 1962). We write this loss rate as 
Eq.~XX. In addition, the star's radiative core and convective envelopes 
exchange angular momentum (Eq.~YY). In Eq.~YY $\tau_c$ is the coupling timescale 
at which the core and envelope are driven to solid-body rotation. 

To find the time-evolution of the stellar quantities in 
Eqs.~XX and~YY, namely 
$R_\star(t),I_\mathrm{rad}(t),I_\mathrm{conv}(t),R_\mathrm{rad}(t),$ 
and $M_\mathrm{rad}(t)$, we compute stellar evolutionary tracks using 
\texttt{MESA} (Paxton et al. 2011, 2013, 2015).
We specifically follow the work of `\texttt{MESA} Isochrones \& Stellar Tracks' 
(MIST)\footnote{Currently hosted at 
\url{http://waps.cfa.harvard.edu/MIST/index.html}}, a project 
that has built and tested large grids of single-star 
evolution models across all evolutionary phases, for a wide range of masses and 
metallicities (Dotter 2016, Choi et al. 2016).

In nearly all cases, we adopt the same physics as Choi et al. 2016, hereafter 
C+16.
We downloaded the \texttt{MESA} input files (\texttt{inlists}) from the MIST 
website\footnote{\url{http://waps.cfa.harvard.edu/MIST/data/tarballs_v1.0/MESA_files.tar.gz},
\texttt{inlists} retrieved 2016-09-20.} and, subject to minor modifications, 
directly used them to compute stellar evolutionary tracks 
across our desired parameter space.
That parameter space is defined as:
\begin{align}
0.4 &\leq M_\star / M_\odot \leq 1.2\quad (0.1M_\odot\ \mathrm{spaced}) 
\label{eq:grid_defn0}\\
-1 &\leq \log_{10} Z/Z_{\odot,\mathrm{protosolar}} \leq 0.5\quad 
(0.25\mathrm{dex\ spaced}),
\label{eq:grid_defn1}
\end{align}
where the grid resolution is primarily limited by convenience (we will 
interpolate between these grids to match them with measured masses and 
metallicities of actual stars), and the range is set by the parameter space of 
observational relevance.
We begin each stellar model in its pre-main sequence\footnote{``Uniform 
composition, $T_c=5\times10^5\,\mathrm{K}$ so no nuclear burning, and uniform 
contraction for enough luminosity to make it fully convective'' 
(\url{http://mesa.sourceforge.net/star_job_defaults.html})}, follow its 
evolution
through its main sequence, and halt it once its He core-mass reaches 
$0.1M_\odot$ (this is often when the star is older than the age of the 
universe, but it is computationally cheap and convenient in later 
interpolation).

Actual jobs were run with \texttt{mesa-r7503}, installed with the appropriate 
\texttt{mesasdk-r7503} and modified following the README of C+16. They were run 
on \texttt{adroit} using \texttt{slurm} arrays for job submission (1 star gets 
1 core was best use of max allocation).

\subsection{Adopted physics}
Table 1 of C+16 summarizes the physics behind our stellar evolution models 
(reproduced for reference as Fig.~\ref{fig:table1_C16}).
We replicate these assumptions exactly except for two points, both detailed 
below: \textit{i)} the solar abundance scale, and \textit{ii)} the specifics of 
transitions between boundary conditions as a function of stellar mass.


\subsubsection{Fine-tuned solar abundance scale}
\begin{figure}[t]
	\centering
	\includegraphics[width=\textwidth]{figs/Choi2016_MIST_table_1.png}
	\caption{Table 1 of C+16. (Purely for reference)}
	\label{fig:table1_C16}
\end{figure}

While C+16 use the protosolar abundances recommended by Asplund et al. 2009 
($Z_{\odot,\mathrm{protosolar}} = 0.0142$), we find, in agreement with the 
solar model discussed by C+16 (Sec. 4), that replicating the bulk properties 
of the Sun at its current age requires adopting fine-tuned protosolar 
abundances.
We insist that our stellar grid reproduce the Sun's radius, luminosity, 
effective temperature to $\lesssim 1\%$ at an age of 4.57 Gyr.
We also insist that our models yield an accurate moment of inertia for the Sun 
(although this is a model-dependent quantity, we can compare our \texttt{MESA} 
profiles with independent models from helioseismology).
For our stellar evolutionary tracks, we adopt the protosolar abundances of
$Y_{\odot,\mathrm{protosolar}} = 0.2612$ and $Z_{\odot,\mathrm{protosolar}} = 
0.0150$, per Table 2 of C+16.\footnote{The tension between the Asplund et al. 
2009 abundances and helioseismology results is to date unresolved, see 
Serenelli 2010 `New results on standard solar models' and Bergemann \& 
Serenelli 2014, `Solar Abundance Problem'.}

In the general case, we then specify the metallicity $\mathrm{[Z/H]=[Fe/H]}$ of 
a given star with respect to $Z_{\odot,\mathrm{protosolar}} = 0.0150$. 
Given $Z$, we find the corresponding hydrogen and helium mass fractions via
\begin{align}
Y_p &= 0.249 
\label{eq:Planck_primordial_He}\\
Y &= Y_p + 
\left(\frac{Y_{\odot,\mathrm{protosolar}}-Y_p}{Z_{\odot,\mathrm{protosolar}}} 
\right) Z \\
X &= 1 - Y - Z,
\end{align}
where the primordial helium abundance $Y_p=0.249$ is adopted from the results 
of the Planck Collaboration (2015).
To find the initial mass fractions of $\mathrm{^1H}$, $\mathrm{^2H}$, 
$\mathrm{^3 He}$, $\mathrm{^4 He}$ (which \texttt{MESA} takes as input), we 
use the isotopic ratios given by Asplund et al. 2009, Table 3.

As a note of warning, $\mathrm{[Z/H]=-1}$ does \textit{not} correspond to a 
metal mass fraction of $0.00150$.
This is because 
\begin{equation}
\mathrm{[Fe/H]} \equiv \mathrm{[Z/H]} = \log_{10} 
\left(\frac{Z}{Z_{\odot,\mathrm{protosolar}}} \right)
-
\log_{10} \left(\frac{X}{X_{\odot,\mathrm{protosolar}}} \right)
\neq
\log_{10} \left(\frac{Z}{Z_{\odot,\mathrm{protosolar}}} \right).
\end{equation}
Given a specific metal mass fraction $Z$, the appropriate metallicity 
$\mathrm{[Z/H]}$ (found by rearranging) is 
\begin{equation}
\mathrm{[Z/H]} = \log_{10} \left(\frac{Z}{Z_{\odot,\mathrm{protosolar}}}\right) 
- \log_{10} \left(\frac{1}{X_{\odot,\mathrm{protosolar}}} \left[1-Y_p - 
Z\left(1 + \frac{Y_{\odot,\mathrm{protosolar}} - 
Y_p}{Z_{\odot,\mathrm{protosolar}}}\right)\right] \right).
\label{eq:metallicity_to_mass_frac}
\end{equation}
With this correction of the hydrogen mass fraction, our grid, defined by 
Eqs.~\ref{eq:grid_defn0} and~\ref{eq:grid_defn1}, has stars with metallicities 
$\mathrm{[Fe/H]}$
of \texttt{[-1.01444556, -0.76321518, -0.51101857, -0.25708473, 0., 0.2628914 
, 0.53680605]}.

\subsubsection{Transitions between boundary conditions}
\label{sec:transitions}
C+16 Sec. 3.3 discusses the various options by which one can constrain the 
temperature and pressure in the outermost cell of a 1D stellar model.
As stated by C+16, the MIST tables (and thus, ours) use a grid 
of 1D plane-parallel atmosphere models computed using ATLAS12 (Kurucz 1970, 
1993). These atmosphere models include molecules that heavily influence the 
opacities of cool dwarfs such as water vapor and titanium monoxide.
With the grid of model atmospheres computed, the standard choice is to set the 
`boundary' at $\tau=2/3$ (the photosphere).
However, the atmospheres of cooler dwarfs are influenced by molecules that are 
not included in \texttt{MESA}'s interior model calculations (Choi et al 2016).
A better choice for cooler dwarfs is thus to set the boundary condition 
at a deeper optical depth, \textit{e.g.}, at $\tau=100$.
C+16 do this by using the \texttt{photosphere\_tables} boundary condition 
for $0.6-10M_\odot$ and the \texttt{tau\_100\_tables} for $0.1-0.3M_\odot$. In 
the intermediate $0.3-0.6M_\odot$, C+16 compute stellar models separately 
using each boundary condition, and then interpolate between them following the 
scheme of Dotter (2016).
We do something similar, but simpler: defined on our desired stellar mass range 
of $0.4-1.2M_\odot$, we say: 
\begin{align}
\mathtt{Boundary\ condition\ table} &=
\begin{cases}
\mathtt{photosphere\_tables} \quad 
&\mathrm{if}\,0.6\leq M_\star/M_\odot<1.2\\
\mathtt{tau\_100\_tables} \quad 
&\mathrm{if}\,0.4\leq M_\star/M_\odot<0.6.
\end{cases}
\end{align}

Comparing the low-mass tracks we get when reproducing the MIST models more
exactly (Sec.~\ref{sec:replicate_MIST}), the difference between our MESA
models, which exclusively use \texttt{tau\_100\_tables}, and those of the MIST
grids, which interpolate between \texttt{tau\_100\_tables} and
\texttt{photosphere\_tables}, is small (see the plots in
\texttt{results/writeup\_plots/MESA\_MIST\_comparison}, and 
Fig.~\ref{fig:MESA_MIST_comparison_0.5Msun}).


\subsection{Comparison to standard solar models}
\label{sec:solar_calibration}
We show our solar calibration in Fig.~\ref{fig:models_near_sun_age}.
`Predicted' values in this plot are: $R_\odot = 
6.957\times10^{10}\,\mathrm{cm}$, $L_\odot = 
3.928\times10^{33}\,\mathrm{erg\,s^{-1}}$, $T_\mathrm{eff,_\odot} = 
5772\,\mathrm{K}$, and $I_\mathrm{tot,\odot} = 
7.0799\times10^{53}\,\mathrm{g\,cm^2} = 
0.0736 M_\odot R_\odot^2$, where $M_\odot = 1.988\times10^{33}\,\mathrm{g}$.
We find the `predicted' solar total moment of inertia by numerically 
integrating the density profile of the standard solar model given by 
Christensen-Dalsgaard et al. 1996.
In Fig.~\ref{fig:models_near_sun_age}, we compute and show 
$I_\mathrm{tot,\star}$ only for our MESA grids because those of the MIST 
project are not published (this was one of the main reasons we opted to make 
our own stellar evolution grids).
The fine-tuned MESA model we use in our grids (solid lines) is `good enough' 
for our purposes: at the Sun's age, we are at $\lesssim 1\%$ agreement with 
predicted values in $R_\star, T_\mathrm{eff,\star}$, and 
$I_\mathrm{\star,tot}$, and $\sim$ 2\% agreement in $L_\star$.
The published `Sun' from C+16's MIST model is further from the expected values 
on all counts\footnote{N.b. the plotted C+16 MIST model is the actual model 
used in the main MIST grids, not the `fine-tuned' solar model discussed in C+16 
Sec. 4. The latter might do better than our MESA-computed model, but is not 
published. We obtain the plotted model by downloading the relevant `EEP track' 
from the 
MIST website, 
\url{http://waps.cfa.harvard.edu/MIST/data/tarballs_v1.0/MIST_v1.0_feh_p0.00_afe_p0.0_vvcrit0.4_EEPS.tar.gz},
and taking the \texttt{00100M.track.eep} model.}.

As an obscure aside, there is no `standard' value for the Sun's moment of 
inertia. The quantity is purely model-determined (a direct measurement would 
require a known torque showing a measurable effect on the Sun), and in a 1D 
stellar model is computed as 
\begin{equation}
I_{\mathrm{tot,\star}} = \frac{8\pi}{3} \int_0^{R_\star} r^4 \rho(r) dr.
\end{equation}
Thus the density profile completely determines the star's total moment of 
inertia. Various values for $I_\mathrm{tot,\star}$ are quoted in the 
literature without specification of the associated model (e.g., in Allen's 
Astrophysical Quantities (2000), or in Table 19, Ch1 of Yoder's chapter of 
Global Earth Physics (1995)). An older `standard' solar model is Table 28.3 of 
Schwarzschild (1958). Other values are given by M. Bursa et al., Studia 
Geophysica et Geodaftica (1996), Iorio, Solar Physics, 1:47, (2012), and Komm 
et al., Astrophysical J., 586, 650, (2003).
As a heavily-cited and somewhat-recent `standard solar model', from this 
hodge-podge I selected the model of Christensen-Dalsgaard et al (1996). 


\begin{figure}[t]
	\centering
	\includegraphics[width=\textwidth]{figs/models_near_sun_age.pdf}
	\caption{Solar calibration (see Sec.~\ref{sec:solar_calibration} for 
	description).}
	\label{fig:models_near_sun_age}
\end{figure}


\subsection{Replicate the MIST profiles}
\label{sec:replicate_MIST}
``But Luke!'', you say, ``you're basically replicating C+16, and 
Fig.~\ref{fig:models_near_sun_age} says you're getting slightly different 
results! How can we be sure that you know what you're doing?''
Answer: by replicating the actual MIST profiles to the best of my ability,
following the description from C+16 and their published \texttt{inlists}.
This gives Figs.~\ref{fig:MESA_MIST_comparison_0.7Msun} 
and~\ref{fig:MESA_MIST_comparison_0.5Msun}.

\begin{figure}[t]
	\centering
  \includegraphics[width=\textwidth]{figs/MESA_MIST_comparison/{0.7Msun_MESA_MIST_comparison}.pdf}
  \caption{The \texttt{Choi\_repl} MESA grids (dotted lines) are my attempt at 
  replicating the  MIST models (solid lines).
  The actual grids I'm arguing we should use are \texttt{production\_0}, which 
  are slightly different. 
  This plot is for $M_\star=0.7M_\odot$, $\mathrm{[Fe/H]=0}$ ($Z=0.015$
  for the \texttt{production\_0} grids, $Z=0.0142$ for \texttt{Choi\_repl}).
  \texttt{Choi\_repl} is pretty close to the published grid from C+16.
  See all the masses in \texttt{/results/writeup\_plots/MESA\_MIST\_comparison}
  }
	\label{fig:MESA_MIST_comparison_0.7Msun}
\end{figure}
\begin{figure}[t]
	\centering
	\includegraphics[width=\textwidth]{figs/MESA_MIST_comparison/{0.5Msun_MESA_MIST_comparison}.pdf}
	\caption{Same as Fig.~\ref{fig:MESA_MIST_comparison_0.7Msun}, but for 
	$0.5M_\odot$. Here I did not reproduce the interpolation scheme described 
	in Sec.~\ref{sec:transitions}, but the \texttt{Choi\_repl} line is still 
	close-ish to C+16 (but not as close of a reproduction as for 
	$M_\star\geq0.6 M_\odot$).
	}
	\label{fig:MESA_MIST_comparison_0.5Msun}
\end{figure}


\subsection{The actual grid in $M_\star$ and $Z$}
Satisfied with our sanity checks, I show the resulting grid of stellar 
evolutionary tracks in Fig.~\ref{fig:grids}.
The relevant tables are available in 
\texttt{/results/grid\_production\_0/tables}.


\newpage
\clearpage
\begin{figure}[t]
	\thispagestyle{empty}
	\centering
	\includegraphics[width=\textwidth]{figs/grids_2by3.pdf}
	\caption{Stellar evolutionary grids from $0.4-1.2M_\odot$ (more massive 
	means bigger radius and $\sim$smaller convective moment of inertia during 
	MS lifetime).
	The thick black lines are $\mathrm{[Fe/H]}=0$ ($Z\equiv Z_\odot = 0.015$) 
	models, 
	varying over mass.
	The thinner gray lines vary $\mathrm{[Fe/H]}\approx-1\ \mathrm{to}\ 0.5$, 
	where fainter means lower metallicity (see 
	Eq.~\ref{eq:metallicity_to_mass_frac} for description of why `$\approx$'). 
	The thin gray lines are plotted for $1.1$ and $0.5M_\odot$.		
	}
	\label{fig:grids}
\end{figure}



\end{document}
